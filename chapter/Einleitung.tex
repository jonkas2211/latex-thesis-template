%!TEX root = ../Thesis.tex
\section{Einleitung}
Geschäftsprozessen werden zunehmend häufiger von \gls{IT} unterstützt. Nicht jedes Unternehmen produziert seine gesamte IT eigenständig. Man denke z.B. an Office von Microsoft. Selbst wenn ein großer Teil der IT im Unternehmen verwaltet und geschaffen wird, erfordert dies ein hohes Maß an Kommunikation zwischen den Beteiligten. Dies kann zu Intransparenz führen. Der Bereich der IT-Governance soll bei diesem Problemen Abhilfe durch Konzepte und Vorschläge verschaffen. \footcite[Vgl.][445]{meyer_it-governance_2003}

IT Abteilungen in Unternehmen wandeln sich zu Dienstleistern innerhalb der Unternehmen. Dabi müssen Sie auch häufig mit der Konkurenz des freien Marktes kämpfen. Deshalb ist die Messung von Leistung an dieser Stelle bedeutsam. Das IT-Governance definiert hier Strukturen durch Regeln für die Zusammenarbeit von IT. \footcite[Vgl.][446\psq]{meyer_it-governance_2003}

Die Zielsetzung dieser Arbeit ist die Einordnung des Begriffes IT-Governance und die Darstellung der Aufgaben von IT-Governance. 
Es wird auch aufgezeigt, welche Möglichkeiten zur Anwendung von IT-Governance in betracht gezogen werden können.
Im Rahmen dieser Arbeit wird nicht aufgezeigt, wie IT-Governance konkret angewendet werden kann. 

