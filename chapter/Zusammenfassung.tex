%!TEX root = ../Thesis.tex
\nopagebreak
\section{Zusammenfassung}

IT-Governance ist ein Teil der Corporate Governance und somit Aufgabe der Unternehmensführung und des gehobenen Managements. Ziel ist die Überwachung und Kontrolle der IT des Unternehmens. Dabei handelt es sich, sowohl um die interne, als auch um die externe IT. 
Die Aufgaben der IT-Governance sind:
\begin{itemize}
	\item Die Ausrichtung der IT-Strategie an der Unternehmensstrategie (IT-Strategic-Alignment)
	\item Der Beitrag der IT zum Unternehmenserfolg durch Generierung von Nutzen
	\item Umgang mit Risiken durch ein Risikomanagement
	\item Sicherstellung der IT-Compliance zu internen und externen Regulierungen und Regeln durch z.B. Gesetze
	\item Messung des Erfolgs der IT (IT-Performance)
	\item Implementierung eines Ressourcenmanagements für den verantwortungsvollen Umgang mit Ressourcen.
\end{itemize}

Die Anwendung von IT-Governance ist individuell für jedes Unternehmen zu gestalten. Implementiert wird es durch Strukturen und Prozesse. Zur Implementation von IT-Governance können Frameworks wie z.B. COBIT oder ITIL genutzt werden. Diese können kombiniert werden und dadurch komplementär wirken. 
